\documentclass[11pt,oneside]{amsart}  
\usepackage[left=0.5in,right=3in,top=1in]{geometry}
\setlength{\marginparwidth}{2.5in}
\usepackage{amsmath}
\usepackage{amsthm}
\usepackage{amsfonts}
\usepackage{amssymb}
\usepackage{eucal}
\usepackage{yfonts}
\usepackage[all]{xy}
\usepackage{amsxtra}
\usepackage{appendix} 
\usepackage{tensor} 
\usepackage{url}
\usepackage{upgreek}
\usepackage{etoolbox}
\usepackage{marginnote}
\usepackage{xcolor}
\apptocmd{\sloppy}{\hbadness 10000\relax}{}{}

\usepackage[pdftex, 
            breaklinks, 
            linktocpage=true, 
            bookmarksopen=true,
            bookmarksopenlevel=0,
            bookmarksnumbered=true]{hyperref}


\RequirePackage{color}
\definecolor{bwgreen}{rgb}{0.183,1,0.5}
\definecolor{bwmagenta}{rgb}{0.7,0.0,0.1}
\definecolor{bwblue}{rgb}{0.317,0.161,1}


\hypersetup{
%pdftitle = {},
pdfauthor = {\textcopyright\ Bryden Cais},
pdfcreator = {\LaTeX\ with package \flqq hyperref\frqq},
colorlinks = {true},
linkcolor={bwmagenta},	 %color red Color for normal internal links.
anchorcolor={},	 %color	black Color for anchor text.
citecolor={bwblue},	% color	green Color for bibliographical citations in text.
filecolor={},	% color cyan Color for URLs which open local files.
menucolor={},	% color	red	Color for Acrobat menu items.
runcolor={},	% color	filecolor Color for run links (launch annotations).
urlcolor={purple}, %	 color	magenta	Color for linked URLs.
} 

%\xymatrixcolsep{2.8pc}
\CompileMatrices
\entrymodifiers={+!!<0pt,\fontdimen22\textfont2>}
%\def\objectstyle{\displaystyle}
%%%%%%%%%%%%%%%%%%%%%%%%%%%%%%%%%%%%%%%
% Introduce a nice script font for sheaves

\DeclareFontFamily{OT1}{rsfs}{}
\DeclareFontShape{OT1}{rsfs}{n}{it}{<-> rsfs10}{}
\DeclareMathAlphabet{\mathscr}{OT1}{rsfs}{n}{it}

\DeclareFontFamily{OT1}{pzc}{}
\DeclareFontShape{OT1}{pzc}{n}{it}{<->s*[2.2]pzc}{}
\DeclareMathAlphabet{\mathpzc}{OT1}{pzc}{b}{sl}

%%%%%%%%%%%%%%%%%%%%%%%%%%%%%%%%%%%%%%%%
% These commands allow one to make roman numerals
\makeatletter
\newcommand{\rmnum}[1]{\romannumeral #1}
\newcommand{\Rmnum}[1]{\expandafter\@slowromancap\romannumeral #1@}
\makeatother

%%%%%%%%%%%%%%%%%%%%%%%%%%%%%%%%%%%%%%%
% Math Operators
\DeclareMathOperator{\Ann}{Ann}
\DeclareMathOperator{\depth}{depth}
\DeclareMathOperator{\zar}{zar}
\DeclareMathOperator{\Cat}{Cat}
\DeclareMathOperator{\Cov}{Cov}
%\DeclareMathOperator{\can}{can}
\DeclareMathOperator{\tot}{tot}
\DeclareMathOperator{\gr}{gr}
\DeclareMathOperator{\id}{id}
\DeclareMathOperator{\spc}{sp}
\DeclareMathOperator{\Frac}{Frac}
\DeclareMathOperator{\ord}{ord} 
\DeclareMathOperator{\nil}{nil}
\DeclareMathOperator{\Sym}{Sym} 
\DeclareMathOperator{\Stab}{Stab} 
%
%\DeclareMathOperator{\pr}{pr}
\newcommand*{\pr}{\rho}
\newcommand*{\ps}{\sigma}
%
\DeclareMathOperator{\Div}{Div} 
\DeclareMathOperator{\Hom}{Hom}
\DeclareMathOperator{\End}{End}
\DeclareMathOperator{\Ext}{Ext}
\DeclareMathOperator{\Gal}{Gal}
\DeclareMathOperator{\GL}{GL}
\DeclareMathOperator{\SL}{SL} 
\DeclareMathOperator{\PSL}{PSL}
\DeclareMathOperator{\PGL}{PGL}
\DeclareMathOperator{\Aut}{Aut}
\DeclareMathOperator{\Spec}{Spec}
\DeclareMathOperator{\Spf}{Spf}
\DeclareMathOperator{\Sp}{Sp}
\DeclareMathOperator{\Proj}{Proj}
\DeclareMathOperator{\coker}{coker}
\DeclareMathOperator{\et}{\acute{e}t}
\DeclareMathOperator{\rig}{rig}
\DeclareMathOperator{\Fr}{Fr}
\DeclareMathOperator{\dR}{dR}
\DeclareMathOperator{\Cris}{Cris}
\DeclareMathOperator{\cris}{cris}
\DeclareMathOperator{\MW}{MW}
\DeclareMathOperator{\tr}{tr}
\DeclareMathOperator{\Tr}{Tr}
\DeclareMathOperator{\an}{an}
\DeclareMathOperator{\Ner}{N\acute{e}r}
\DeclareMathOperator{\Pic}{Pic}
\DeclareMathOperator{\Alb}{Alb}
\DeclareMathOperator{\sm}{sm}
\DeclareMathOperator{\Extrig}{Extrig}
\DeclareMathOperator{\Lie}{Lie}
\DeclareMathOperator{\Inf}{Inf}
%\DeclareMathOperator{\BT}{BT}
\DeclareMathOperator{\reg}{reg}
\DeclareMathOperator{\Ab}{Ab}
\DeclareMathOperator{\tors}{tors}
\DeclareMathOperator{\free}{fr}
\DeclareMathOperator{\red}{red}
\DeclareMathOperator{\smooth}{sm}
\DeclareMathOperator{\Char}{char}
\DeclareMathOperator{\nr}{nr}
\DeclareMathOperator{\Mod}{Mod}
\DeclareMathOperator{\Fil}{Fil}
\DeclareMathOperator{\MF}{MF}
\DeclareMathOperator{\rk}{rk}
\DeclareMathOperator{\st}{st}
\DeclareMathOperator{\Rep}{Rep}
\DeclareMathOperator{\tor}{tor}
\DeclareMathOperator{\sep}{sep}
\DeclareMathOperator{\perf}{rad}
\DeclareMathOperator{\un}{un}
\DeclareMathOperator{\projdim}{projdim}
\DeclareMathOperator{\res}{res}
\DeclareMathOperator{\Def}{Def}
\DeclareMathOperator{\Ht}{ht}
\DeclareMathOperator{\GD}{GD}
\DeclareMathOperator{\Ig}{Ig}
\DeclareMathOperator{\len}{length}
\DeclareMathOperator{\cone}{cone}
\DeclareMathOperator{\supp}{supp}
\DeclareMathOperator{\im}{im}
\DeclareMathOperator{\rad}{rad}
\DeclareMathOperator{\Con}{Conn}
\DeclareMathOperator{\Irr}{Irr}
\DeclareMathOperator{\bal}{bal.}
\DeclareMathOperator{\Ell}{Ell}
\DeclareMathOperator{\Sets}{Sets}
\DeclareMathOperator{\univ}{univ}
\DeclareMathOperator{\MC}{M}
\DeclareMathOperator{\Rad}{Rad}
\DeclareMathOperator{\abs}{abs}
\DeclareMathOperator{\Cot}{Cot}
\DeclareMathOperator{\pdiv}{pdiv}
\DeclareMathOperator{\ModFI}{ModFI}
\DeclareMathOperator{\inv}{IF}
\DeclareMathOperator{\mult}{m}
\DeclareMathOperator{\loc}{ll}
\DeclareMathOperator{\Jac}{Jac}
\DeclareMathOperator{\Frob}{F}
\DeclareMathOperator{\Ver}{V}
\DeclareMathOperator{\proj}{proj}
\DeclareMathOperator{\incl}{incl}
\DeclareMathOperator{\Hodge}{Hdg}
\DeclareMathOperator{\ssimp}{ss}
\DeclareMathOperator{\Null}{null}
\DeclareMathOperator{\sh}{sh}
\DeclareMathOperator{\PD}{PD}
\DeclareMathOperator{\Sen}{Sen}
\DeclareMathOperator{\cusps}{cusps}
%%%%%%%%%%%%%%%%%%%%%%%%%%%%%%%%%%%%%%%%%%%%%%%%%%%%%%%%%%%%%%%%%%%%%%%%%%%%%%%
%New Commands		

%%% The usual rings %%%
\newcommand*{\R}{\ensuremath{\mathbf{R}}}   
\renewcommand*{\c}{\ensuremath{\mathbf{C}}}              
\newcommand*{\Z}{\ensuremath{\mathbf{Z}}}               
\newcommand*{\Q}{\ensuremath{\mathbf{Q}}}                           
\newcommand*{\N}{\ensuremath{\mathbf{N}}}             
\newcommand*{\Qbar}{\overline{\Q}}                             
\renewcommand*{\k}{k}             
\newcommand*{\kbar}{\overline{k}}       
\newcommand*{\Kbar}{\overline{K}}    
          
%%% The usual schemes %%%
\renewcommand*{\P}{\ensuremath{\mathbf{P}}}           		
\newcommand*{\Gm}{\ensuremath{{\mathbf{G}_m}}}   
\newcommand*{\Ga}{\ensuremath{{\mathbf{G}_a}}}   

%%% Gothic letters (ideals) %%%
\newcommand*{\m}{\mathfrak{M}}                          
\newcommand*{\n}{\mathfrak{N}}                            
\newcommand*{\p}{\mathfrak{p}}                            
\newcommand*{\q}{\mathfrak{q}}                            
\renewcommand*{\r}{\mathfrak{r}}       
\newcommand*{\s}{\mathfrak{S}}

%%% Capital scripts (sheaves etc)
\newcommand*{\A}{\ensuremath{\mathcal{A}}}
\newcommand*{\scrA}{\mathscr{A}}
\newcommand*{\B}{\mathcal{B}}
\newcommand*{\scrC}{\mathscr{C}}
\newcommand*{\C}{\mathbf{C}}
\newcommand*{\E}{\mathscr{E}}     
\newcommand*{\F}{\mathbf{F}}
\newcommand*{\scrF}{\mathscr{F}}
\newcommand*{\calF}{\mathcal{F}}
\newcommand*{\scrS}{\mathscr{S}}
\newcommand*{\scrW}{\mathscr{W}}
\newcommand*{\G}{\mathcal{G}}
\newcommand*{\scrG}{\mathscr{G}}  
\newcommand*{\scrH}{\mathscr{H}}                           
\newcommand*{\h}{\mathscr{H}}                               
\newcommand*{\I}{\mathscr{I}}                               
\newcommand*{\J}{\mathcal{J}}
\renewcommand*{\L}{\mathscr{L}}
\newcommand*{\scrM}{\mathscr{M}}
\newcommand*{\scrT}{\mathscr{T}}
\newcommand*{\calM}{\mathcal{M}}
\renewcommand*{\O}{\mathscr{O}}                    
\newcommand*{\U}{\mathscr{U}} 
\newcommand*{\X}{\mathcal{X}}     
\newcommand*{\Y}{\mathcal{Y}}

%%% script capitals %%%
\newcommand*{\scrJ}{\mathscr{J}}                               
\newcommand*{\calL}{\mathcal{L}}                              
\newcommand*{\calS}{\mathcal{S}}                             
\newcommand*{\scrP}{\mathscr{P}}  
\newcommand*{\scrQ}{\mathscr{Q}}                             

%%% Various sheaves %%%
\newcommand*{\scrHom}{\mathscr{H}\mathit{om}}      
\newcommand*{\scrExtrig}{\mathscr{E}\mathit{xtrig}}	
\newcommand*{\scrExt}{\mathscr{E}\mathit{xt}}               
\newcommand*{\scrLie}{\mathscr{L}\mathit{ie}}      
\newcommand*{\scrD}{\mathscr{D}}          

%%% Misc %%%
\newcommand*{\EE}{\mathbb E}        
\newcommand*{\barrho}{\overline{\rho}}                        
\newcommand*{\invlim}{\varprojlim}                               
\newcommand*{\tensorhat}{\widehat{\otimes}}             
\newcommand*{\D}{\ensuremath{\mathbf{D}}}
\newcommand*{\M}{\ensuremath{\mathbf{M}}}
\newcommand*{\calD}{\ensuremath{\mathcal{D}}}              
\renewcommand*{\H}{\ensuremath{\mathbf{H}}}
\newcommand*{\z}{\zeta} 
\newcommand*{\dual}{\vee}
\renewcommand*{\qedsymbol}{\ensuremath{\blacksquare}}            % end of proof
\newcommand*{\eqdef}{\stackrel{\text{def}}{=}}     % definition
\newcommand*{\wa}[1]{{}^{\mathrm{w.a.}}\hspace{-0.4ex}{#1}}
\newcommand*{\mfun}{\underline{\M}}
\newcommand*{\Mfun}{\underline{\m}}
\newcommand*{\Dfun}{\underline{D}}
\newcommand*{\Gfun}{\underline{G}}
\newcommand*{\Dual}[1]{{{#1}^t}}
\newcommand*{\VDual}[1]{{{#1}^{\vee}}}
%\newcommand*{\R}{\ensuremath{\mathscr{R}}}              
\renewcommand*{\b}{\ensuremath{\mathrm{b}}}
\renewcommand*{\int}{\ensuremath{\mathrm{int}}}
\newcommand*{\bnd}{\ensuremath{\mathrm{bnd}}}
\newcommand*{\e}{\ensuremath{\mathbf{E}}}
\renewcommand*{\a}{\ensuremath{\mathbf{A}}}
\newcommand*{\Dst}{\ensuremath{{\underline{D}_{\st}}}}
\renewcommand*{\div}{\ensuremath{\mathrm{div}}}
\renewcommand*{\SS}{\ensuremath{{\mathrm{ss}}}}
\renewcommand*{\u}[1]{\underline{#1}}
\renewcommand*{\o}[1]{\overline{#1}}
\newcommand*{\wh}[1]{\widehat{#1}}
\newcommand*{\wt}[1]{\widetilde{#1}}
\newcommand*{\nor}[1]{{#1}^{\mathrm{n}}}
\newcommand*{\T}{\ensuremath{\mathbf{T}}}
\newcommand*{\tens}{\mathop{\otimes}\limits}
\newcommand*{\can}{\text{-}\mathrm{can}}
\newcommand*{\fiber}{\mathop{\times}\limits}
\newcommand*{\subscript}[1]{\ensuremath{_{#1}}}

\DeclareMathOperator{\Win}{Win}
\DeclareMathOperator{\BT}{BT}

%%%%%%%%%%%%%%%%%%%%%%%%%%%%%%%%%%%%%%%%%%%%%%%%%%%%%%%%%%%%%%%%%%%%%%%%%%%%%%%
% Define environments

\theoremstyle{plain}
  \newtheorem{theorem}{Theorem}
  \newtheorem{proposition}[theorem]{Proposition}
  \newtheorem{lemma}[theorem]{Lemma}
  \newtheorem{corollary}[theorem]{Corollary}

\theoremstyle{definition}
  \newtheorem{definition}[theorem]{Definition}
  \newtheorem{notation}[theorem]{Notation}
  \newtheorem{hypotheses}[theorem]{Hypotheses}
  \newtheorem{question}[theorem]{Question}
  \newtheorem{philosophy}[theorem]{Philosophy}	

\theoremstyle{remark}
  \newtheorem{example}[theorem]{Example}
  \newtheorem{remark}[theorem]{Remark}
  \newtheorem{remarks}[theorem]{Remarks}
  \newtheorem{exercise}[theorem]{Exercise}
  \newtheorem{warning}[theorem]{Warning}
  \newtheorem{convention}[theorem]{Convention}	
\include{header}
\numberwithin{theorem}{section}  
\numberwithin{equation}{section}

\newenvironment{dotlist}
   {
      \begin{list}
         {$\cdot$}
         {
            \setlength{\itemsep}{.5ex}
            \setlength{\parsep}{0ex}
            \setlength{\leftmargin}{1em}
            \setlength{\parskip}{0ex}
            \setlength{\topsep}{0ex}
         }
   }
   {
      \end{list}
   }

%%%%%%%%%%%%%%%%%%%%%%%%%%%%%%%%%%%%%%%%%%%%%%%%%%%%%%%%%%%%%%%%%%%%%%%%%%%%%%%%

% This comment creates colored and named commentary for use with collaborators


%\usepackage[usenames,dvipsnames]{color}

\newcommand{\bryden}[1]{{\color{red} \sf $\heartsuit\heartsuit\heartsuit$ Bryden: [#1]}}




%%%%%%%%%%%%%%%%%%%%%%%%%%%%%%%%%%%%%%%%%%%%%%%%%%%%%%%%%%%%%%%%%%%%%%%%%%%%%%%

%This command creates a footnote and an indicator mark in the margin pointing to problems, questions, or unresolved (small) issues
%To use type \edit{ insert text here }

\newcommand{\marginalfootnote}[1]{%
        \footnote{#1}
        \marginpar[\hfill{\sf\thefootnote}]{{\sf\thefootnote}}}
\newcommand{\edit}[1]{\marginalfootnote{#1}}


%%%%%%%%%%%%%%%%%%%%%%%%%%%%%%%%%%%%%%%%%%%%%%%%%%%%%%%%%%%%%%%%%%%%%%%%%%%%%%%

%This command creates a box marked ``To Do'' around text.
%To use type \todo{  insert text here  }.

\newcommand{\todo}[1]{\vspace{5 mm}\par \noindent
\marginpar{\textsc{ToDo}}
\framebox{\begin{minipage}[c]{0.95 \textwidth}
\tt #1 \end{minipage}}\vspace{5 mm}\par}

%%%%%%%%%%%%%%%%%%%%%%%%%%%%%%%%%%%%%%%%%%%%%%%%%%%%%%%%%%%%%%%%%%%%%%%%%%%%%%%

\usepackage[normalem]{ulem}
\newcommand\hl{\bgroup\markoverwith
    {\textcolor{yellow}{\rule[-.5ex]{.1pt}{2.5ex}}}\ULon}

\begin{document}
\title{Science History Podcast: \emph{Numbers and Number Systems}}
\subjclass[2010]{}
\keywords{}
\date{\today}

\maketitle

\section{Introduction}
\marginnote{\footnotesize This document serves as informal reference material for the Science History Podcast Episode 43 on Number Theory; and was drafted by Bryden Cais, Max von Hippel, and Frank von Hippel.}

\noindent In this podcast episode, we will discuss the history and development of Number 
Theory, viewed through the lens of numbers and number systems.
We will begin with the \emph{natural numbers}, originally
\[\N:=\{1,2,3,\ldots\}\]
\marginnote{\footnotesize Alternatively denoted $\mathbb{N}$.}
which arose to abstractly represent and manipulate tallies.
Over the course of thousands of years, this system of counting numbers was 
studied, expanded, and generalized to include zero, 
\marginnote{\footnotesize
  The Greeks mostly ignored 0 because they used
  geometric lengths (e.g. of string) rather than symbolic numbers.
  The Pythagoreans thought of numbers as multitudes of 1s,
    and apparently did not consider 1 itself to be a number, 
    let alone zero~\cite{neg.numbers}.
  There are two kinds of 0:
    the number representing \emph{nothing},
    as in $1 - 1 = \text{\hl{0}}$, and
    a \emph{placeholder} for multiplication, as in the decimal 
    $12\text{\hl{0}}6 = ( (12) \text{\hl{$\times$(100)}} ) + 6$.
  The earliest known placeholder zero
    is on a Babylonian tablet from $\approx400$BCE,
  that uses a base-60 number system with $''$ for a placeholder.
  Ptolemy used a placeholder 0 in 130CE, but only as punctuation.
  The Mayans discovered the placeholder 0 by 665CE,
  while the Indians began using it by 876CE.
  By 830CE the Indians had discovered the number 0, but they did not
  understand the impossibility of division-by-0 until more than 500 years
  later.  
  Fibonacci, an Italian mathematician, used both kinds of 0, but called 0
  a ``sign'' rather than a number.
  Al-Khw{\-a}rizm{\-i}, the namesake of ``algorithm'' and ``algebra'', 
  wrote about the Hindu 0 in Iraq in the 12$^{\text{th}}$ century,
  as did Qin Jiushao in China in 1247.~\cite{history.of.zero}
}
negative numbers,
rationals, real numbers, irrational, algebraic and transcendental numbers, 
complex numbers, modular arithmetic, and $p$-adic numbers.  
All of these systems of numbers play an important role in {\em Number Theory} 
which, at its core, is the study of $\N$ and its properties.  
The goals of this episode are:
\begin{enumerate}
	\item Provide a mathematical and historical overview of these different number 
        systems and their development.
	\item Highlight the challenges and obstacles that mathematicians and 
        civilizations faced with new concepts of \emph{number}.
	\item Give the listener a sense of why these various systems of numbers are 
        interesting and important.
	\item Touch on some of the important unsolved problems in modern number 
        theory, and how these different number systems play a role.
\end{enumerate} 
A good overview reference for some of the history is 
  \href{https://en.wikipedia.org/wiki/Number}{Wikipedia}.
Hopefully this document and the notes in its margins are also useful.

\section{The integers}
Actually, there is some debate about the definition of $\N$.  
Many texts define $\N$ as above, 
  but I would argue that the ``right" definition should include zero, as below.
\[\N:=\{0,1,2,3,\ldots\}\]
There are a few good arguments for this: 
  the natural numbers are the ``counting" numbers, 
  that is, they are exactly the numbers which
  are cardinalities of finite sets
\marginnote{\footnotesize
The Russell-Zermelo paradox (discovered around 1903)
considered the ``set'' of all sets that do not contain
themselves.  This strange set is apparently a member of itself, 
if and only if it is not a member of itself.
In the reasoning systems ZF (Zermelo Fraenkel) and ZFC (with Choice) 
the paradox disappears, because the word ``set'' is not allowed to include
such pathological examples.~\cite{russel.paradox}
} 
  (modern mathematics is predicated on axiomatic set theory, 
  typically using the ZF or ZFC axioms).
In set theory, one of the axioms is that the intersection of two sets should 
  always be a set, and that necessitates having the {\em empty set}, 
  $\emptyset$,
  which contains no elements and has cardinality $0$.  
  So if $\N$ is the set of cardinalities of finite sets, it must include zero.

Another argument has to do with arithmetic: the natural numbers are closed under 
addition and multiplication, and the element $1$ is a 
{\em multiplicative identity} as it satisfies $1\cdot n=n\cdot 1=n$ for every 
natural $n$.  From that point of view, it makes sense that $\N$ should also 
contain an {\em additive identity}, which satisfies $x + n = n + x = n$ for 
all naturals $n$.  Of course, $x=0$ is the unique value that works!

This second argument will return as a central, key theme: 
\begin{philosophy}
	Mathematics often isolates some property that we would like a number to 
  satisfy, and if we don't already have any number to satisfy it, we {\em make} 
  a new number and larger number system to verify the property. 
\end{philosophy}

Such is the genesis of the integers
\marginnote{\footnotesize Alternatively denoted $\mathbb{Z}$.}
\[\Z:=\{\ldots -3,-2,-1,0,1,2,3,\ldots\},\]
consisting of the naturals (with zero!) and their negatives.
\marginnote{\footnotesize The first recorded use of negative numbers was in \emph{The Nine Chapters on the Mathematical Art}, a textbook written by multiple generations of Chinese mathematicians between 100BCE and 50CE, in which negatives were denoted with black dots, and positives with red dots.  In the 3$^{\text{rd}}$ century CE, Diophantus' book \emph{Arithmetica} described certain equations as ``absurd'' because they yielded negative solutions.  In the 7$^{\text{th}}$ century CE, the Indian mathematician Brahmagupta extended Diophantus' work to treat negative numbers as numbers in their own right.  The Europeans did not catch up to Brahmagupta in their philosophical respect for negative numbers for another $\approx$700 years.  The French mathematician Chuquet introduced negative exponents in the 15$^{\text{th}}$ century CE, calling them ``absurd'' exponents.  Descartes, Pascal, and Leibniz all thought negative numbers absurd, even if propositionally valid.  Maclaurin and Euler justified negative numbers using the metaphor of debt.  However, controversy around the apparent absurdity of negative and imaginary numbers persisted in European literature through the 18$^{\text{th}}$ century.~\cite{neg.numbers}}

It took civilization some time to accept negative quantities as ``numbers" in 
their own right, but the above philosophy plays a critical role here.  
Addition is fundamental to operations of the number system $\N$,
so questions like ``is there a number $x$ for which $x+3=5$" arise naturally.  
Of course, in this case $x=2$ is the unique solution.
But when we reverse the roles of $3$ and $5$, as in the equation $x+5=3$, 
suddenly there is {\em no solution} if we only allow ourselves to 
work in the natural numbers!  So a solution is {\em created}, and called $-2$, 
and it satisfies the {\em defining} attribute that $2 + (-2) = 0$.

I don't think that negative integers {\em exist} in the same way that positive integers do: you can {\em show} me $1$ apple, you can hold it in your hand,
you can give it to me and I can take a bite out of it.  But you can't do any of that with $-1$ apples, and one has to invoke notions of debt (already a most abstract notion!) to make ``real world" sense of negative numbers. One of the central points I'd like to make is that mathematics is {\em best}---most complete, powerful, useful, beautiful---when we don't worry so much about ``real world" meaning or application of mathematical ideas or theories.  That is a counter-intuitive idea, 
as the real world has informed and spurred so much important mathematics, but the mathematics itself develops best untethered from these shackles of its origins.

Now the integers are a lot better than the naturals, because they form a {\em ring}.  A ring is a set of numbers in which one can do arithmetic
(addition and multiplication) {\em and} there are both additive and multiplicative identities, {\em and} every element has an additive inverse,
meaning that we can always solve the equation $x+r = 0$ for any number $r$.  One also asks that the usual axioms (associativity, distributivity of
multiplication over addition etc.) hold.  
\marginnote{\footnotesize  The Irish mathematician  William Rowan Hamilton discovered the Quaternions on October 16$^{\text{th}}$, 1843.  The Quaternions were the first important non-commutative ring discovered.  Hamilton's goal was to extend the Complex numbers to a structure with one real part and two imaginary parts, a Theory of Triplets, to be used for representing rotations in 3D.  However, no such Theory exists~\cite{hankins1977triplets}.  
Instead, he one day found and immediately etched into the Broome Bridge a Theory of Quartets: Quaternions~\cite{ring.history,buchmann2011brief}.
This illustrates how one cannot always just invent a number or number system to satisfy some interesting requirements, because sometimes (as in the Algebra of Triplets) the requirements are collectively unsatisfiable.}
For the purposes of this discussion, we'll only talk about {\em commutative} rings, 
{\em i.e.} rings in which
$$x\cdot y = y\cdot x$$
for all $x$ and $y$.  There are lots of important examples of non-commutative rings, like rings of square matrices, but that is a topic for another time.


\section{The rationals}

Being a ring, the integers are closed under addition and multiplication, and every integer has an additive inverse, but very few integers
have {\em multiplicative} inverses.  Said differently, we can solve equations like $4x = 12$ because 12 is an integer multiple of 4, 
but change that 12 to a 10 and we are out of luck!  So it is reasonable to create a new number system in which every equation $mx=n$
with integers $m$ and $n$ has a solution.  Actually, this leads to a degenerate number system, because having a solution to $0\cdot x = 1$
causes problems, and many of the desirable properties of a ``number system" break down.  For example, if $x$ satisfies the above equation,
so does $x+n$ for any integer $n$, so either there are infinitely many $x$'s that solve the equation, or we have to accept that $x=x+n$ for all $n$,
and either we aren't allowed to subtract $x$ from both sides (ruining basic arithmetic), or in our new system of numbers $n=0$ for all $n$,
and that is not very useful!  So by {\em definition}, the rationals are all solutions $x$ to equations $mx=n$ with $m\neq 0$.  We
represent such a solution as the fraction $n/m$, but then we have some intricate rules to maintain the usual features of arithmetic,
since if $mx=n$ then $2mx = 2n$, so $2n/2m = n/m$.  
\[
	\Q:=\left\{\frac{n}{m}\ :\ n,m\in \Z, m\neq 0\right\}
\]
It took a long time to really understand fractions, because strictly speaking a {\em fraction} like $3/5$ is an 
\marginnote{\footnotesize Although hinted at since the dawn of mathematics, equivalence classes were first formally introduced in Jourdain's 1912 paper \emph{On isoid relations and theories of irrational number}.  Jourdain explained how the word ``equivalence'' had been previously used by Cantor to describe the cardinal numbers, and he suggested that the same word could be used to describe numbers that were inherently related in other ways.  In 1926, Hasse explained how a partition of a set yields ``classes'', thus alluding to the idea of equivalence classes, although he did not call them such.  Van der Waerden’s \emph{Moderne Algebra} explicitly discussed the \emph{{\"A}quivalenzrelation} (equivalence relation), but the term did not survive translation to English. In 1935, Birkhoff illuminated the relationship between equivalence classes and relations, but he called the classes ``categories''.  In 1942, Oystein Ore called them ``blocks''.  The term ``equivalence class'' became more commonplace in the late 30s, but was not defined generally, only in the context of a given relation (i.e. for the relation $R$, one might discuss the $R$-class, denoted $[R]$ in the style of Lefschetz).  Finally, equivalence relations were defined foundationally in Tukey’s \emph{Convergence and Uniformity in Topology} (1940)~\cite{asghari2019equivalence}.  Today equivalence has found new footing as a foundational concept in Homotopy Type Theory, where one of the axioms (the Univalence Axiom) says ``equivalence is equivalent to equality''~\cite{awodey2013voevodsky}.  In that case, the relevant kind of equivalence is homotopical.}
equivalence class of pairs of integers
\[\begin{aligned}
	\frac{3}{5} 
    & = \{ (n,m)\ :\ n,m\in \Z,\ m\neq 0,\ 3m=5n \} \\
    & = \{ (3,5), (6,10), (9,15), (-3,-5), (-6,-10),\ldots\}
\end{aligned}
\]
This is just saying that the fractions $\frac{3}{5}, \frac{6}{10}, \frac{9}{15}, \frac{-3}{-5},\ldots$ are all the {\em same} number, because they solve mathematically equivalent
equations.  But they don't teach equivalence classes in school when we learn about fractions, and I think this is a difficult and subtle concept.

The set of rational numbers forms what is called a {\em field}: it is a (nonzero) ring in which every nonzero number has a multiplicative inverse.
Fields have the pleasing property that for any elements $a,b,c$, the linear equation $ax+b=c$ always has a solution, at least as long
as $a\neq 0$.

The advent of the decimal system allows us to represent rationals as decimals, like $\frac{3}{5} = 0.6$ and $\frac{1}{3} = 0.\overline{3}$,
and there is a rather nice characterization of which decimal expansions are rational numbers: 

\begin{proposition}\label{rat}
	A decimal expansion comes from a rational number if and only if it either terminates, or is eventually repeating.
\end{proposition}

One confounding feature of decimals is that, just like a rational number can have many representations as a fraction ($\frac{3}{5} =\frac{6}{10}= \frac{9}{15}$),
so too a number can have more than representation as a decimal expansion!  A classic example is:
\[
  1.\overline{0} = 0.\overline{9}.
\]
This example is really frustrating for many people, and has resulted in more than one shouting match and even tears. 
For whatever reason, many people {\em believe} that every number has a unique decimal expansion, possibly because
we are taught from an early age to equate the concepts of {\em number} and {\em decimal expansion}.  But it just isn't true,
much in the same way that rational numbers don't have a unique representation as a fraction.  In grade-school, one of my math teachers
{\em insisted} that $1$ and $0.\overline{9}$ are {\em different} numbers, and no amount of reasoning would convince them otherwise.  But if you
admit that decimal expansions represent numbers, and therefore must obey the basic laws of arithmetic (these are axiomatized!), then
\marginnote{\footnotesize This argument is intuitively convincing, but formal arguments are more complicated.  The classic argument taught to mathematics students uses the $(\epsilon, \delta)$-definition of a \emph{limit}, and can be unsatisfying because, at the introductory stage of a real-analysis course, it is not yet obvious to the students that such limits actually define equality.  Another approach is a proof based on infinite series.  We know that for all $r$, if $|r|<1$, then: 

\(\begin{aligned}
\sum_{i = 1}^{\infty} ar^i = \frac{ar}{1 - r}
\end{aligned}\)

Plugging in $r=\frac{1}{10}$ and $a=9$ gives

\(\begin{aligned}0.\overline{9} & = \sum_{i = 1}^{\infty} 9(\frac{1}{10})^i \\
& = \frac{9(\frac{1}{10})}{1 - \frac{1}{10}} \\
& = \frac{9/10}{9/10} \\
& = 1 \end{aligned}\)

This proof appears in Euler's 1770 \emph{Elements of Algebra}~\cite{euler.elements}.
}
$1=0.\overline{9}$ is forced upon you, since:
\[
\frac{1}{3} = 0.\overline{3}
\]
by long division, so multiplying both sides by 3 yields $1=0.\overline{9}$.  
Said another way, whatever number $x=0.\overline{9}$
is, it has to obey the rules of arithmetic, so has to satisfy $10x - x = 9$ since multiplication by 10 shifts the decimal to the right by 1, 
and subtracting $x$ from this lops off the infinite repeating tail of $9's$.  But then $9x=9$, which again forces $x=1$.

\section{Real numbers}

With Proposition \ref{rat} in mind, it is natural to ask what numbers are represented by {\em arbitrary} decimal expansions.
These are of course the real numbers: $\R$, and they solve a basic problem with the rationals, which is that the rational numbers have {\em holes}.
That is, there are infinite sequences of rational numbers which are getting arbitrarily close to {\em something}, but that something isn't a rational number!
For example:
\[
	3, 3.1, 3.14, 3.141, 3.1415, \cdots
\]
This just reflects the fact that if you truncate an arbitrary decimal expansion at any point, you will get a terminating decimal expansion, 
which must therefore be a rational number.  
From this point of view, the rational numbers are {\em not complete} with respect to the usual measure of distance given by absolute value:
\marginnote{\footnotesize This distance function is a \emph{metric}.  The French mathematician Maurice Fr{\'e}chet initiated the study of metrics and metric spaces (spaces topologized by metrics) in 1905~\cite{frechet}.  Today metric spaces are fundamental to operations research, machine learning, algorithm design, and numerous other disciplines.
}
\[
	d(x,y) := |x - y|.
\]
The reason is that there are lots of sequences of rationals as above, with the property that the distance between any two terms
in the sequence gets smaller and smaller as you go out, but there is no ``final term", or limit, of the sequence within the system of 
rational numbers.  In this way, the real numbers are the {\em completion} of the rationals with respect to the usual distance function above:
they ``fill in'' all of the holes that the usual distance function perceives in the form of sequences whose terms get increasingly close together.
Such sequences are called {\em Cauchy sequences}, after the French mathematician Augustin--Louis Cauchy (1789--1857), who
was a pioneer of mathematical rigor in algebra and analysis,
 and introduced the much-loathed (by students) notion of $(\epsilon,\delta)$
arguments to Calculus.

The system of real numbers has many wonderful properties, including {\em continuity}, and therefore provides the right setting
in which to do calculus.  But they have some rather bizarre and frustrating features: there are continuous functions that are differentiable 
nowhere, nonzero smooth functions which vanish identically on any closed subset, nonnegative integrable functions with zero integral, and so forth.

% \todo{Discussion of irrational, algebraic and transcendental numbers?}

\section{$p$-Adic numbers}
As explained above, the real numbers are created by ``filling in the holes" of the rational numbers that are detected by the
usual distance function $d(x,y)=|x-y|$.  
But this is {\em not} the only distance function on the rational numbers, and
different distance functions will pick up on different ``holes", which
\marginnote{
  \footnotesize $p$-Adic numbers were first explicitly discovered by the German
  mathematician Kurt Wilhelm Sebastian Hensel in 1897~\cite{hensel1897neue}.
  However, the translator's introduction to Dedekind and Weber's 
  \emph{Theory of algebraic functions of one variable}
  notes: ``Indeed, with hindsight it becomes apparent that a discrete valuation 
  is behind Kummer's concept of ideal numbers.''~\cite{dedekind2012theory}
  The implication is that Kummer---in this case, the German mathematician Ernst Eduard Kummer---may 
  have implicitly used $p$-adic numbers before Hensel did.
  This does not exactly mean that Kummer came up with the $p$-adic numbers first;
      he might not have known his work had such deeper significance.
  The mathematician Christopher Zeeman once debated the historian David Fowler
  for 40 years about weather or not Euclid viewed ratios as something like an equivalence
  class~\cite{asghari2019equivalence}.  
  The question is similar to asking if Kummer really \emph{knew} that he was doing $p$-adic
  number theory.  Zeeman wrote, in correspondence to Fowler, ``The historian thinks extrinsically 
  in terms of the written evidence and adheres strictly to that data, whereas the mathematician 
  thinks intrinsically in terms of the mathematics itself, which he freely rewrites in his own notation 
  in order to better understand it and to speculate on what might have been passing through the mind of the ancient mathematician, without bothering to check the rest of the data.''~\cite{zeeman2008s}
}
 are filled in by entirely new systems of numbers!
To motivate this, which will seem very strange at first given our rich experience of the real numbers, consider the rational
number 0.  This is arguably the {\em smallest} rational number by any sense of small since adding it to any other rational number
does nothing.  But there is another {\em multiplicative} measure of smallness that zero satisfies: it is the only integer
that is {\em arbitrarily divisible}, in the sense that $0$ is a multiple of $n$ for every integer $n$.  Thinking in terms of 
decimals, $0$ is the only integer which, when represented in base 10, has arbitrarily many zeros to the left of the decimal point.
Said in terms of arithmetic, 0 is the only integer which is divisible by $10^k$ for every natural number $k$.
This inspires a new prototype of distance:

\begin{definition}
	The 10-adic absolute value of an integer $n$ is $1/10^k$, where $10^k$ is the highest power of
	10 dividing $n$.
\end{definition}  

By definition, the 10-adic absolute value of $0$ is 0.  This notion of size 
is really different from the usual one, since $1000$ is smaller than $100$, which is smaller than $10$,
and $311$ and $7$ have the same size!  But actually, 10 is a bad choice, because we don't really get a well-behaved
distance function from it.  Indeed, the 10-adic absolute values of $2$ and $5$ are both 1, but their product has 
$10$-adic absolute value $0.1$, so this new absolute value isn't multiplicative.  That causes all kinds of problems,
and it doesn't really deserve to be called an absolute value.  But this problem is caused by the fact that 10 can be factored as $2\cdot 5$,
so the problem goes away if we replace 10 by a prime number $p$.  In this way, we get for every prime $p$,
a $p$-adic absolute value on $\Z$, which can be extended to $\Q$ by declaring that the absolute value of
a ratio is the ratio of the absolute values of numerator and denominator:
\begin{align*}
	|n|_p &:= 1/p^k\ \text{where}\ p^k\ \text{is the highest power of}\ p\ \text{dividing}\ n \\
	\left|\frac{m}{n}\right|_p &:= \frac{|m|_p}{ |n|_p}
\end{align*}

For example, using the $3$-adic absolute value, the sequence $3,9,27,81,243,\ldots$ is approaching 0,
while the sequence $2,4,8,16,32,\cdots$ is a sequence of numbers all of the same $3$-adic absolute value $1$.
So for each prime $p$, we get a new distance function on the rational numbers: $d_p(x,y)=|x-y|_p$,
and these new distance functions ``see" new holes in the rational numbers, which can be filled in (like we made the real numbers)
to produce the {\em $p$-adic numbers $\Q_p$}, which is a completely new number system for each prime $p$.
Said differently, $\Q_p$ is the completion of $\Q$ with respect to the $p$-adic absolute value.
We now have infinitely many different ways of filling in the holes in the rationals:
$$
  \R, \Q_2, \Q_3, \Q_5, \Q_7, \Q_{11}, \ldots
$$
and an amazing theorem of Alexander Ostrowski (1893--1986) asserts that these are the {\em only} completions of $\Q$;
{\em i.e.} that there are no other ways to fill in the holes.

The number systems $\Q_p$ are really different from $\R$, largely because the $p$-adic distance function is {\em ultrametric}
in the following sense.
The usual absolute value, which completes the rationals to the reals, satisfies the {\em triangle inequality}: 
the length of the hypotenuse of a triangle is at most the sum of the lengths of the other two sides.  But the $p$-adic
absolute value satisfies the {\em strong triangle inequality}, in which the hypotenuse of a triangle has length at most the
maximum of the other two sides!  No sum needed!  This leads to the remarkable fact that the collection of all
$p$-adic numbers of $p$-adic absolute value at most 1 forms a ring, so is a number system in its own right:
$$
	\Z_p:=\{x\in \Q_p \ :\ |x|_p \le 1\}
$$
In fact, the $p$-adic absolute value sees ``holes" already in $\Z$, and $\Z_p$ is what you get when you fill in these holes.
For example, the sequence
$$
	1, 4, 13, 175, 4549, 11110\ldots
$$
converges to a limit $L$ in $\Z_3$ because the successive differences are $3,3^2,2\cdot 3^4, 2\cdot 3^7, 3^8\ldots$,
which have $3$-adic absolute values tending to zero.  Actually, the $3$-adic limit of this sequence $L$ satisfies $L^2 = 7$,
so that $\sqrt{7}$ exists in $\Z_3$; this can be seen by looking at $a^2 - 7$ for each $a$ in the sequence above, and we get the
integers
$$
	-2\cdot 3, 3^2, 2\cdot 3^4, 2\cdot 7\cdot 3^7,2\cdot19\cdot 83\cdot 3^8,\ldots
$$
which are progressively more and more divisible by $3$, so $3$-adically are tending to zero.


\section{What are the $p$-Adic numbers good for?}

Arithmetic---addition and multiplication---lead (as we have seen) to 
algebraic equations like $nx=m$ and $ax+b=c$, and eventually to more complex  equations
like
\begin{align*}
	&x^2 + y^2 = z^2
\end{align*}
Broadly construed, (algebraic) number theory is the study of (systems of) algebraic equations and their solutions.  
Solving equations like the above with $x,y,z$ {\em real numbers} is easy!  We can pick more or less
any values for $x$ and $y$ that we like, and then just solve for $z$.  This leads to lots of real solutions, like
$$
(x,y,z) = (1,1,\sqrt{2}), (1,2,\sqrt{5}), (2,3,\sqrt{13}),\cdots
$$
but it is really hard (or even impossible with many systems) to find any nontrivial\footnote{
The solutions $(x,y,z)=(\pm 1, 0,\pm 1), (0,\pm1, \pm 1)$ are called {\em trivial}
becuase they are rather obvious, and not as interesting as solutions where $x$, $y$, and $z$ are all nonzero.
}
{\em rational} or {\em integer solutions}.
Of course, the system above is a really famous one, whose integer solutions are exactly the {\em Pythagorean triples},
like $(3,4,5)$ and $(5,12,13)$.  It is important to emphasize that computers are not very helpful in finding such integer solutions:
because there are infinitely many integers, we can't have a computer just plug in numbers and hope that it will eventually stumble upon an integer solution,
since such a procedure might never finish, and even if there are integer solutions to be found, they are very rare compared to the real solutions,
so there isn't much chance to find them by happenstance.  

It turns out that for any system of algebraic equations like the one above, 
finding solutions in $\Q_p$ for each prime $p$, or determining that none exist, 
is about as easy as finding solutions in real numbers.  Roughly speaking, this is because $\Q_p$---like the real numbers---is complete
with respect to an absolute value, so doesn't have any ``holes", and we can more or less plug in randomly and then solve for the missing values.
This doesn't always work, just as it doesn't always work in the reals (for example, $x^2+y^2+z^2 = -1$ has no real solutions at all).
But finding $p$-adic solutions to systems of equations, or determining that none exist, is a tractable problem,
just as it is in the reals (for example, one has a $p$-adic Newton's method for producing solutions from approximations by an iterative procedure).

This same problem for {\em integer} solutions is called {\em Hilbert's tenth problem}, and was formulated by the influential 
German mathematician David Hilbert in his now eponymous list of problems, presented to the International Congress of Mathematicians in 1900.
In 1970, building on prior work of many mathematicians, Matiyasevich finally proved in 1970 that Hilbert's tenth problem is {\em unsolvable},
meaning that there can be no algorithm to decide if a general system of algebraic equations has an integer solution or not.
Amazingly, Matiyasevich's proof uses Fibonacci numbers in an essential way!

Nonetheless, an amazing result of Hasse and Minkowski {\em does} provide an algorithm for equations of small degree:

\begin{theorem}
	If an equation in finitely many variables with rational coefficients involves only the pairwise products of
	the variables,
	then it has a rational solution if and only if it has a solution in $\R$ and in every $\Q_p$.
\end{theorem}

So, for example, the Pythagorean equation above has terms only of degree $2$, so the fact that it has solutions in $\R$
and in every $\Q_p$ implies that it has rational (end even integer) solutions.

This idea---that we can solve algebraic equations in rational numbers if we can do so in every completion 
of the rationals---is called the {\em Hasse principle}, and is a central philosophy in number theory.  Unfortunately, it doesn't always hold!
For example, Selmer found the equation
$$
	3x^3 + 4y^3 + 5z^3 = 0
$$
has nontrivial solutions in $\R$ and in every $\Q_p$, but no nontrivial rational solutions at all!




 







\bibliographystyle{alpha}
\bibliography{podcast}


\end{document}